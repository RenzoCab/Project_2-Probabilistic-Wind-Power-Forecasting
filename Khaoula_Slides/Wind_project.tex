\documentclass{beamer}

\mode<presentation>
{
  \usetheme{default}     
  \usecolortheme{default} 
  \usefonttheme{default} 
} 

\usepackage[french]{babel}
\usepackage[utf8]{inputenc}
\usepackage[T1]{fontenc}
\usepackage{graphicx}
\usepackage{pgffor}
\usepackage{tabularx}
\usepackage{adjustbox}
\usepackage{booktabs}
\usepackage{subfig}
\usepackage{ragged2e}


\apptocmd{\frame}{}{\justifying}{}

\setbeamertemplate{caption}[numbered]
\setbeamertemplate{frametitle}{\color{blue}\textbf{\small\insertframetitle}}
%\setbeamerfont{frametitle}{size=\small}
\setbeamertemplate{footline}{% 
  \hfill% 
  \usebeamercolor[fg]{page number in head/foot}% 
  \usebeamerfont{page number in head/foot}% 
  \insertframenumber%
  %\,/\,\inserttotalframenumber
  \kern1em\vskip2pt% 
}
\graphicspath{{C:/Users/HP/Desktop/plots/}}}

}

\title[Title]{Wind project inference}
\author{Khaoula Ben Chaabane}
\institute{Ecole Polytechnique de Tunisie}
\date{02/06/2020}


\begin{document}

\begin{frame}
  \titlepage
  \thispagestyle{empty}
\end{frame}

\section{Introduction}
\begin{frame}{Introduction}
\justifying
In these slides, I will explain the inference process with which we estimate the parameters of our model.
\end{frame}

\section{Model}
\begin{frame}{Model}
\justifying
The wind power production is modeled as follows, where $X_{t}$ is the normalized real production:$$\\ 
\left\{\begin{array}{l}
d X_{t}=\left(\dot{p}_{t}-\theta_{t}\left(X_{t}-p_{t}\right)\right) d t+\sqrt{2 \alpha \theta_{0} X_{t}\left(1-X_{t}\right)} d W_{t}, \quad t \in[0, T] \\
X_{0}=x_{0} \in[0,1]
\end{array}\right.$$$$\\
We may introduce the following model for the forcecast error of the normalized wind power production where $X_{t}$ is the real production, $p_{t}$ the forecast and $V_{t}=X_{t}-p_{t}$  is the error:
$$
\begin{equation}
\left\{\begin{array}{l}
d V_{t}=-\theta_{t} V_{t} d t+\sqrt{2 \alpha \theta_{0}\left(V_{t}+p_{t}\right)\left(1-V_{t}-p_{t}\right)} d W_{t}, \quad t \in[0, T] \\
V_{0}=v_{0} \in[-1+\varepsilon, 1-\varepsilon]
\end{array}\right.
\end{equation}
\end{frame}

\begin{frame}{Model}
To guarantee a unique solution for the process $X_{t},$ $\theta_{t}$ needs to be bounded for $t \in[0, T]$. We have that:
$$$$
\theta_{t}=\max \left(\theta_{0}, \frac{\alpha \theta_{0}+\left|\dot{p}_{t}\right|}{ \min \left(1-p_{t}, p_{t}\right)}\right)
$$$$
This is not true for $\theta_{t}$ if $p_{t} \rightarrow 0^{+}$ or $p_{t} \rightarrow 1^{-} .$ Therefore we need to ensure that $p_{t} \in[\varepsilon, 1-\varepsilon]$ for some $0<\varepsilon<\frac{1}{2}$, $\forall$ $t \in[0, T]$.
\end{frame}

\begin{frame}{Model}
We define then the corrected forecast:
$$
p_{t}^{\varepsilon}=\left\{\begin{array}{ll}
\varepsilon & \text { if } \quad p_{t}<\varepsilon \\
p_{t} & \text { if } \quad \varepsilon \leq p_{t}<1-\varepsilon \\
1-\varepsilon & \text { if } \quad p_{t} \geq 1-\varepsilon
\end{array}\right.
$$\\
and the corrected (and bounded) drift coefficient is therefore:
$$
\theta_{t}^{\varepsilon}=\max \left(\theta_{0}, \frac{\alpha \theta_{0}+\left|\dot{p}_{t}^{\varepsilon}\right|}{ \min \left(1-p_{t}^{\varepsilon}, p_{t}^{\varepsilon}\right)}\right)
$$
\end{frame}

\section{Likelihood}

\begin{frame}{Likelihood}
\justifying
We sample each of our M continuous-time Itô process $V=(V_{t})_{t \in[0, T]}$  at $N+1$ equidistant discrete points with a given length interval $\Delta$.
\\ $V^{M, N+1}=\left\{V_{t_{1}}^{N+1}, V_{t_{2}}^{N+1}, \ldots, V_{t_{M+1}}^{N+1}\right\}$ denotes this random sample, with $V_{t_{j}}^{N+1}=\left\{V_{t_{j}+i \Delta}, i=0, \ldots, N\right\}, \forall j\in\left\{1, \ldots, M\right\}.$
\vskip 0.5cm
\\Let $\rho\left(v | v_{j, i-1} ; \boldsymbol{\theta}\right)$ be the conditional probability density of $V_{t_{j}+i \Delta} \equiv V_{j, i}$,  given $V_{j, i-1}$  where $\boldsymbol{\theta}=\left(\theta_{0}, \alpha\right)$ are the unknown model parameters.
\vskip 0.5cm
The Itô process $V$ defined by the SDE (1) is Markovian, then the likelihood function of the sample $V^{M, N+1}$ can be written as follows:
\[
\mathcal{L}\left(\boldsymbol{\theta} ; V^{M, N+1}\right)=\prod_{j=1}^{M}\left\{\prod_{i=1}^{N} \rho\left(V_{j, i} | V_{j, i-1} ; p_{\left[t_{j, i-1}, t_{j, i}\right]}, \boldsymbol{\theta}\right)\right\}
\]
where $t_{j, i} \equiv t_{j}+i \Delta$ for any $j\in\left\{1, \ldots, M\right\}$ and $i\in\left\{0, \ldots, N\right\}$
\end{frame}

\begin{frame}{Likelihood approximation}
\justifying
In order to compute the exact likelihood function, we need a closed-form expression of the transition probability of V which can be found using the Fokker-Planck equation:
\vskip 0.5cm
 \begin{array}{l}
\frac{\partial f}{\partial t} \rho\left(v, t | v_{j, i-1}, t_{j, i-1} ; \theta\right)=-\frac{\partial}{\partial v}\left(-\theta_{t} v \rho\left(v, t | v_{j, i-1}, t_{j, i-1} ; \theta\right)\right) \\
\quad+\frac{1}{2} \frac{\partial^{2}}{\partial v^{2}}\left(2 \theta_{0} \alpha\left(v+p_{t}\right)\left(1-v-p_{t}\right) \rho\left(v, t | v_{j, i-1}, t_{j, i-1} ; \theta\right)\right)
\end{array}\vskip 0.5cm
However, solving this equation is not always possible and is computationally costly. For this reason, we approximated the likelihood using a proxy distribution for V. \\In our case we used a Beta distribution $\Beta(\xi_{1},\xi_{2})$ as for the family of diffusion term in our SDE (1) (Pearson diffusion), it has been proved to be the best approximation. In order to find the parameters $(\xi_{1},\xi_{2})$ of this proxy distribution we will match its first and second moments with the ones of the exact distribution deduced from the SDE (1).
\end{frame}

\begin{frame}{Moment matching: First moment}
\justifying
For a time s $\in \left[t_{n},t_{n+1}\right]$ the exact first moment $m_{1}(s)$ deduced from the SDE (1) is the solution of the following ODE:\\
\left\{\begin{array}{l}
\mathrm{d} m_{1}(s)=\left[-m_{1}(s) \theta(s)\right] \mathrm{d} s \\
m_{1}\left(t_{n-1}\right)=v_{t_{n-1}}
\end{array}\right.
\vskip 0.5cm
We want to compute $m_{1}(t_{n})$:
\begin{itemize}
\item If $\theta(t_{n})=\theta(t_{n+1})=\theta$ then the exact solution is : $m_{1}(t_{n})=m_{1}(t_{n-1})\exp(-\theta(t_{n}-t_{n-1}))$
\item else, we compute a linear approximation of $\theta(s)$ and approximate the ODE using Forward-Euler:
\vskip 0.5cm
$m_{1}\left(s_{n}\right)=m_{1}\left(s_{n-1}\right)\left(1-\theta\left(s_{n-1}\right) \Delta s\right)$
\end{itemize}
\end{frame}

\begin{frame}{Moment matching: Second moment}
\justifying
Using Ito's formula, we find, for a time s $\in \left[t_{n},t_{n+1}\right]$, the exact second moment $m_{2}(s)$ deduced from the SDE (1) is the solution of the following ODE:
\vskip 0.5cm
$\left\{\begin{array}{ll}
\mathrm{d}m_{2}(s)=\left[-2 m_{2}(s)\left(\theta(s)+\alpha\theta_{0}\right)+2 \alpha\theta_{0} m_{1}(s)(1-2 p(s))\\+2\alpha \theta_{0}p(s)(1-p(s))\right]\mathrm{d}s\\
m_{2}\left(t_{n-1}\right)=v_{t_{n-1}}^{2}
\end{array}$.
\vskip 0.5cm
We compute a linear interpolation for the functions $\theta(s)$ and $p(s)$. After, we solve the ODE using Forward-Euler:\\
$m_{2}\left(s_{n}\right)=m_{2}\left(s_{n-1}\right)+\left[-2 m_{2}\left(s_{n-1}\right)\left(\theta\left(s_{n-1}\right)+\alpha \theta_{0}\right)\\+2 \alpha \theta_{0} m_{1}\left(s_{n-1}\right)\left(1-2 p\left(s_{n-1}\right)\right)+2 \alpha \theta_{0} p\left(s_{n-1}\right)\left(1-p\left(s_{n-1}\right)\right)\right] \Delta s
$\vskip 0.5cm
We use the same discretization points for both $m_{1}(s)$ and $m_{2}(s)$.
\end{frame}

\begin{frame}{Moment matching}
\justifying
V is approximated by a new proxy random variable: $V=a+(b-a) X$ with support in $[a,b]=[-1,1],$ where $X \sim \beta\left(\xi_{1}, \xi_{2}\right)$ and $\operatorname{PDF} f_{V}(v)$.
We find the two first moments:
\vskip 0.5cm
\begin{itemize}
\item $\mathbb{E}[V]=a+(b-a) \mathbb{E}[X]=a+(b-a) \frac{\xi_{1}}{\xi_{1}+\xi_{2}}=\mu_{V}$
\item $\mathbb{V}[V]=(b-a)^{2} \mathbb{V}[X]=\frac{(b-a)^{2} \xi_{1} \xi_{2}}{\left(\xi_{1}+\xi_{2}\right)^{2}\left(\xi_{1}+\xi_{2}+1\right)}=\sigma_{V}^{2}$
\end{itemize}
\vskip 0.5cm
We want the first two moments of the true random variable and its approximation to be equal $\forall t$. \\Therefore, $\mu(t)=m_{1}(t)$ and $\sigma^{2}(t)=m_{2}(t)-m_{1}^{2}(t)$. 
\vskip 0.5cm
For each measurement $V_{t_{n-1}},$ we can find the analytical moments at time $t_{n}$ solving the ODEs from the previous slides. We can then find the parameters $\xi_{1}$ and $\xi_{2}$ of the proxy.
\end{frame}

\begin{frame}{Evaluation of $(\xi_{1},\xi_{2})$}
\begin{itemize}
\item $\xi_{1}=-\frac{(1+\mu)\left(\mu^{2}+\sigma^{2}-1\right)}{2 \sigma^{2}}$.
\item $\xi_{2}=\frac{(\mu-1)\left(\mu^{2}+\sigma^{2}-1\right)}{2 \sigma^{2}}$.
\end{itemize}
\vskip 0.5cm
 all evaluated at time $t_{n}$
\end{frame}

\begin{frame}{Log-density of the proxy random variable V}
\justifying
We want to compute the PDF $f_{V}(v)$ of the random variable:\\ $V=a+(b-a)X$. 
\vskip 0.5cm
 For $[a, b]=[-1,1],$ we have that: \\
$f_{V}(v)=f_{X}\left(g^{-1}(v)\right)\left|\frac{\mathrm{d}}{\mathrm{d} v} g^{-1}(v)\right| \quad$ where $\quad f_{X}(x)=\operatorname{Beta}\left(\xi_{1}, \xi_{2}\right) \quad$ and $\quad g(x)=a+(b-a) x$.
\vskip 0.5cm
Then, 
\\$f_{V}(v)=\frac{1}{|(b-a)|} \frac{1}{B\left(\xi_{1}, \xi_{2}\right)}\left(\frac{v-a}{b-a}\right)^{\xi_{1}-1}\left(1-\frac{v-a}{b-a}\right)^{\xi_{2}-1}$, because \\$g^{-1}(v)=\frac{v-a}{b-a}$.\vskip 0.5cm
Therefore:\\
$\log \left(f_{V}(v)\right)=\log \left(\frac{1}{B\left(\xi_{1}, \xi_{2}\right)}\right)+\left(\xi_{1}-1\right) \log \left(\frac{v-a}{b-a}\right)+\left(\xi_{2}-1\right) \log \left(\frac{b-v}{b-a}\right)$
\end{frame}
\begin{frame}{Log-likelihood}
\justifying
We introduce the number of paths (days) $M,$ and the number of measurements per path $N+1(N$ transitions). We have a total of $M \times N$ samples. 	The log-likelihood is:
\[
\mathfrak{L}\left(\{V\}_{M, N}\right)=\sum_{i=1}^{M} \sum_{j=2}^{N+1} \log \left[\rho_{i, j}\left(V_{i, j} | V_{i, j-1}\right)\right]
\]
where $\rho_{i, j}\left(V_{i, j} | V_{i, j-1}\right)=\rho_{i, j}\left(V_{i, j} | V_{i, j-1} ; \xi_{1, j}, \xi_{2, j}\right)$.
\end{frame}


\section{Initial Estimation of the parameters}

\begin{frame}{Initial Estimation of the parameters}
\justifying
In order to evaluate the initial parameters of our model we apply the least square method on the forecast error $V_{t}$.\\ We consider the transition $\Delta V_{i}=V_{i+1}-V_{i}$ with $\Delta t=t_{i+1}-t_{i}.\\
\left(V_{i+1} | V_{i}\right)$ is a random variable which conditional mean can be approximated by the solution of the following system:
\vskip 0.5cm
\left\{\begin{array}{l}
\mathrm{d} \mathbb{E}[V]=-\theta_{t}^{\varepsilon} \mathbb{E}[V] \mathrm{d} t \\
\mathbb{E}\left[V\left(t_{i}\right)\right]=V_{i}
\end{array}\right.

evaluated in $t_{i+1}$ (i.e., $\mathbb{E}\left[V\left(t_{i+1}\right)\right]$ ).
\end{frame}
\section{LSM}
\begin{frame}{Least Square Minimization: LSM}
\justifying
 Then, the random variable $\left(V_{i+1}-\mathbb{E}\left[V\left(t_{i+1}\right)\right]\right)$ has a mean equal to 0 approximately.\\
If we assume that $\theta_{t}^{\varepsilon}=c \in \mathbb{R}^{+}$ for all $t \in\left[t_{i}, t_{i+1}\right],$ then $\mathbb{E}\left[V\left(t_{i+1}\right)\right]=V_{i} e^{-c \Delta t} .$ \\If we have a total of $n$ transitions, we can write the regression problem for the conditional mean with $L^{2}$ loss function as:\\$$
\begin{equation}
%\begin{aligned}
c^{*} \approx \arg \min _{c \geq 0}\left[\sum_{i=1}^{n}\left(V_{i+1}-\mathbb{E}\left[V\left(t_{i+1}\right)\right]\right)^{2}\right]\\&$$=\arg \min _{c \geq 0}\left[\sum_{i=1}^{n}\left(V_{i+1}-V_{i} e^{-c \Delta t}\right)^{2}\right]
%\end{aligned}
\end{equation}
\end{frame}

\begin{frame}{Least Square Minimization: LSM}
\justifying
We take the first order approximation of $e^{-c \Delta t}$  w.r.t. $c$: 
$$
e^{-c \Delta t}=1-c \Delta t+\mathscr{O}\left((c \Delta t)^{2}\right)
$$
and introduce it in equation $(1) .$ We get
$$
c^{*} \approx \arg \min _{c \geq 0} \underbrace{\left[\sum_{i=1}^{n}\left(V_{i+1}-V_{i}(1-c \Delta t)\right)^{2}\right]}_{=f(c)}
$$
As $f(c)$ is convex in $c,$ solving (5) (finding $c^{*}$ ) is equivalent to solving
$$
\frac{\partial f}{\partial c}\left(c^{* }\right)=0
$$
\end{frame}

\begin{frame}{Least Square Minimization: LSM}
\justifying
$$
\begin{aligned}
\frac{\partial f}{\partial c} &=\sum_{i=1}^{n} 2\left(-V_{i}\right)(-\Delta t)\left(V_{i+1}-V_{i}\left(1-\theta_{0} \Delta t\right)\right) \\
&=\sum_{i=1}^{n} 2 V_{i} \Delta t\left(V_{i+1}-V_{i}(1-c \Delta t)\right) \\
&=\sum_{i=1}^{n} 2 V_{i+1} V_{i} \Delta t-2 V_{i}^{2} \Delta t+2 V_{i}^{2} \Delta t^{2} c
\end{aligned}
$$
Then, $c^{* }$ satisfies the following: 
$$
c^{ *} \approx \frac{\sum_{i=1}^{n} V_{i}\left(V_{i}-V_{i+1}\right)}{\Delta t \cdot \sum_{i=1}^{n}\left(V_{i}\right)^{2}}
$$
\end{frame}

\section{Quadratic variation}
\begin{frame}{Quadratic variation}
\justifying
We approximate the SDE by its E-M scheme. In particular, we approximate the Itô quadratic variation with the discrete one:
\begin{itemize}
\item ltô process quadratic variation: $[V]_{t}=\int_{0}^{t} \sigma_{s}^{2} \mathrm{d} s$
\item Discrete process quadratic variation: $[V]_{t}=\Sigma_{0<s \leq t}\left(\Delta V_{s}\right)^{2}$
\end{itemize}$$$$\\
Then, considering $\Delta t$ the time between the measurements, we approximate:
$$
\theta_{0}^{*} \alpha^{*} \approx \frac{\sum_{i=1}^{n}\left(\Delta V_{i}\right)^{2}}{2 \Delta t \sum_{i=1}^{n}\left(V_{i}+p_{i}\right)\left(1-V_{i}-p_{i}\right)}
$$
\end{frame}

\section{\text { Estimation of} $\left(\theta_{0}, \alpha, \varepsilon\right)$}
\begin{frame}{\text { Estimation of} $\left(\theta_{0}, \alpha, \varepsilon\right)$}
\justifying
In this section, we will use the approximation made previously to estimate the parameters $\left(\theta_{0}, \alpha, \varepsilon\right)$ of the SDE. Let us define $\left(\theta_{0}^{*}, \alpha^{*}, \varepsilon^{*}\right)$ as their estimators.$$\\
\text{If we fix} $\varepsilon,$ we define the forecast error $\forall i \in {1...n}$ $V_{i}=X_{i}-p_{i}^{\varepsilon}$. \\ 
If we also fix $\theta_{0}$ and $\alpha,$ we can define the set of indexes: $$ \\
$\mathbf{I}=\left\{i \in\{1, \ldots, n\}: \text { the LSM estimation will estimate } \theta_{0}\right\}$ \\ $\mathbf{J}=\left\{j \in\{1, \ldots, n\}: \text { the } L S M \text { estimation will estimate } \frac{\theta_{0} \alpha}{\varepsilon}\right\}$ \\
We will proceed then to approximate these sets in order to estimate our parameters. 
\end{frame}

\begin{frame}{\text { Estimation of} $\left(\theta_{0}, \alpha, \varepsilon\right)$}
\justifying
To use the LSM estimation, we assumed that $\theta_{t}^{\varepsilon}=c \in \mathbb{R}^{+}, $ and we defined $\theta_{t}^{\varepsilon}$:
$$
\theta_{t}^{\varepsilon}=\max \left(\theta_{0}, \frac{\alpha \theta_{0}+\left|\dot{p}_{t}^{\varepsilon}\right|}{\min \left(1-p_{t}^{\varepsilon}, p_{t}^{\varepsilon}\right)}\right)
$$
From the definition of $\theta_{t}^{\varepsilon}:$ We have that for $\varepsilon<<1,$ and $p_{t}=\varepsilon$ or $p_{t}=1-\varepsilon,$ the approximation $\theta_{t}^{\varepsilon} \approx \frac{\theta_{0} \alpha}{\varepsilon}$ holds. Then, for $\varepsilon$ small enough, $\mathrm{J}$ can be approximated by the following:
$$
\mathbf{J} \approx \mathbf{J}=\left\{j \in\{1, \ldots, n\}: p_{j}^{\varepsilon} \in\{\varepsilon, 1-\varepsilon\}\right\}
$$
We have that it is more likely that $\theta_{t}^{\varepsilon}=\theta_{0}$ if $p_{t}^{\varepsilon} \approx \frac{1}{2} .$ Then, we can approximate I by
$$
\mathbf{I} \approx \tilde{\mathbf{I}}=\left\{i \in\{1, \ldots, n\}: p_{i} \in(\gamma, 1-\gamma)\right\}, \gamma \approx \frac{1}{2}, \gamma<\frac{1}{2}
$$
\end{frame}


 \begin{frame}{Estimation of $\alpha^{*}$}
\justifying
\\ With the previous approximation made of the quadratic variation we can estimate $\color{red} \theta_{0}*\alpha*=0.094$ therefore, with our given estimation of $\theta_{0}*$ we find that: $\color{red} \alpha*=0.08$
\end{frame}
 
 \begin{frame}{Estimation of $\varepsilon^{*}$}
\justifying
Now that we have an approximated value of $\theta_{0} \alpha,$ if we can estimate $\frac{\theta_{0} \alpha}{\varepsilon},$ then we can estimate $\varepsilon$.
We showed previously that for $\varepsilon<<1$, the LSM estimation using indexes from $\mathbf{J}$ is an estimator for $\frac{\theta_{0} \alpha}{\varepsilon}=: k$

The goal is to find values for $\varepsilon$ that satisfy $\varepsilon<<1$. For that we start by randomly choosing a small initial value for $\varepsilon$ (that we will call $\varepsilon_{0}$ ), and iterating we aim to converge to some local minimum. We proceed with the following steps:
\begin{itemize}
\justifying
\item We sample $\varepsilon_{0}$ from $\mathscr{U}[0.01,0.1]$ and load $\varepsilon \leftarrow \varepsilon_{0}$
\item We create $\tilde{J}$ and use the LSM estimation to find $k$. 
\begin{itemize}
\item If $k<\theta_{0}^{*},$ then the assumption $\theta_{t}^{\varepsilon}=c \in \mathbb{R}^{+}$ is wrong and we reduce the value of $\varepsilon,$ i.e., $\varepsilon \leftarrow \varepsilon * 0.999 .$ 
\item If $k \geq \theta_{0}^{*},$ we load $\varepsilon \leftarrow \frac{\theta_{0}^{*} \alpha^{*}}{k}(\text { we allow a maximum relative change of } 1 \%) .$
\end{itemize}
 We repeat this step 100 times.
\item We repeat steps 1 and 2, 50 times.
\end{itemize}
 \end{frame}	
 
 \begin{frame}{Initial parameters estimation}
\justifying
 To conclude, the estimations of the SDE parameters that we found are: $\left(\theta_{0}^{*}, \alpha^{*}, \varepsilon^{*}\right)=(1.25,0.08,0.018).$
 \vskip 0.5cm
 The code computing this process can be found in the file\\  \textbf{Wind\_}\textbf{project\_}\textbf{intial\_}\textbf{guess.ipynb}.
 \end{frame}
 
 \begin{frame}{Log-likelihood optimization}
\justifying
 I minimized the negative log-likelihood using the function \textbold{fmin} from the library \textbold{scipy.optimize} in Python to estimate the parameters $(\theta_{0},\alpha)$. I found the following results for the different datasets provided:\\
 \begin{table}
\centering
\begin{tabular}{|l|l|l|l|} 
\hline
\textbf{Data providers} & $\theta_{0}$ & $\alpha$ & $\theta_{0}\alpha$  \\ 
\hline
First dataset                & 1.161                            & 0.0718                    & 0.083                                                  \\ 
\hline
UTEP5                   & 1.357                            & 0.0809                    & 0.108                                                  \\ 
\hline
MTLOG                   & 1.175                            & 0.0856                    & 0.100                                                  \\ 
\hline
AWSTEP                  & 1.196                            & 0.0846                    & 0.101                                                  \\
\hline
\end{tabular}
\end{table}
 \vskip 0.5cm
 The code of this optimization can be found in the file\\ \textbf{Wind\_project\_optimization.ipynb}
 \end{frame}
 
 \section{$\delta$}
 \begin{frame}{Time parameter $\delta$}
 In order to have $E(X)=p$ at all times, our SDE (1) needs to have a starting point where $V=0$ $(X=p)$.\\ \vspace{0.5cm}
 However, we noticed that with our data, the error $V$ at $t=0$ is not 0. Therefore, to still unsure we have this property, we suppose that there existed a time $t_{\delta}<0$ in the past where the condition $X=p$ was verified. (This can be justified as we assume that the forecast providers had at some point perfect information on the data).  \\ \vspace{0.5cm}.
 \end{frame}
 
 \begin{frame}{Time parameter $\delta$}
 To do this, we will linearly interpolate the forecast to a time $t_{\delta}$ $(t_{0}-t_{\delta}=\delta)$, and for the transition $V^{i}_{t_{0}}|V^{i}_{t_{\delta}}$ $\forall i\in \{1,\dots, M\}$  we will solve the following problem: 
\[
\delta \approx \arg \min _{\delta} \mathcal{L}_{\delta}\left(\boldsymbol{\theta}, \delta ; V^{M, 1}\right)=\arg \min _{\delta} \prod_{j=1}^{M} \rho_{0}\left(V^{i}_{t_{0}} |V^{i}_{t_{\delta}} ; \boldsymbol{\theta}, \delta\right) \]\\
We will approximate again the density with a $\Beta$ distribution and redo the same steps as before of moment matching to find its parameters.\\ \vspace{0.5cm}

We finally find that $\delta=0.0837=120$ minutes.

 \end{frame}
 
 \begin{frame}{Path simulation}
 I plotted the path for the real production, the forecast and some simulations of the production using the model.
\begin{figure}[ht] 
\centering
    \centering
    \includegraphics[width=1\linewidth]{paths0.eps} 
    \label{fig:dynamics}
 \end{figure}
 \end{frame}
 \begin{figure}[ht] 
\centering
    \centering
    \includegraphics[width=1\linewidth]{paths45.eps} 
    \label{fig:dynamics}
 \end{figure}
 \end{frame}
 \begin{figure}[ht] 
\centering
    \centering
    \includegraphics[width=1\linewidth]{paths80.eps} 
    \label{fig:dynamics}
 \end{figure}
 \end{frame}
 \begin{figure}[ht] 
\centering
    \centering
    \includegraphics[width=1\linewidth]{paths105.eps} 
    \label{fig:dynamics}
 \end{figure}
 \end{frame}
 
 \begin{frame}{Confidence intervals}
 I plotted the 99\%, 90\% and 50\% confidence intervals using 100 simulations per day.
\begin{figure}[ht] 
\centering
    \centering
    \includegraphics[width=1\linewidth]{bands0.eps} 
    \label{fig:dynamics}
 \end{figure}
 \end{frame}
 \begin{figure}[ht] 
\centering
    \centering
    \includegraphics[width=1\linewidth]{bands45.eps} 
    \label{fig:dynamics}
 \end{figure}
 \end{frame}
 \begin{figure}[ht] 
\centering
    \centering
    \includegraphics[width=1\linewidth]{bands80.eps} 
    \label{fig:dynamics}
 \end{figure}
 \end{frame}
 \begin{figure}[ht] 
\centering
    \centering
    \includegraphics[width=1\linewidth]{bands105.eps} 
    \label{fig:dynamics}
 \end{figure}
 \end{frame}
 
 

\end{document}

