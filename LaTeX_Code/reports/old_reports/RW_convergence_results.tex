\documentclass[aspectratio=169]{beamer}\usepackage[utf8]{inputenc}
\usepackage[english]{babel}
\usepackage{color}
\usepackage{amsmath,mathtools}
\usepackage{mathptmx}
\usepackage[11pt]{moresize}
\setbeamertemplate{navigation symbols}{}
\setbeamersize{text margin left=5mm,text margin right=5mm}
\usepackage{wrapfig}
\usepackage{bbm}
\usepackage{xcolor}
\usepackage{tabularx}
\usepackage{bm}


\newcommand{\R}{\mathbb{R}}
\newcommand{\E}{\mathbb{E}}
\newcommand{\N}{\mathbb{N}}
\newcommand{\Z}{\mathbb{Z}}
\newcommand{\V}{\mathbb{V}}
\newcommand{\Q}{\mathbb{Q}}
\newcommand{\K}{\mathbb{K}}
\newcommand{\C}{\mathbb{C}}
\newcommand{\T}{\mathbb{T}}
\newcommand{\I}{\mathbb{I}}

\setbeamertemplate{caption}[numbered]

\title{Uncertainty of Power Generation Forecasts}
\subtitle{ Waleed Alhaddad}

\begin{document}
\setbeamercolor{background canvas}{bg=blue!3}
\begin{frame}
\titlepage
\end{frame}



\begin{frame}[label=guide]\frametitle{ Reader's Guide }

List of changes in this iteration:
\begin{itemize}
\item Extrapolate one hour backwards to circumvent the big error of the initial point.
\item Implemented the new parameterized drift term  to prevent the process from leaving the permissible range.
\end{itemize}

Next steps:
\begin{itemize}
\item Find some metric to measure how good is the model
\item Compute the confidence intervals in another way.
\end{itemize}

Note :
\begin{itemize}
\item Green slides: possible future extensions
\item Red slides: notes to be removed
\end{itemize}

\end{frame}


\begin{frame}\frametitle{ Base Model }
Let $V_t$ be the deviation of the wind power generation forecasts from actual wind power generation. That is, $V_t$ models the errors or uncertainty of a given set of wind power generation forecasts. 

We propose a forecast-error model based on the following parameterized stochastic differential equation (SDE), 
\begin{equation}
\begin{split}
dV_t &= a(V_t; \bm{\theta}) dt + b (V_t; \bm{\theta} ) dW_t \quad t > 0 \\
V_0 & = v_0
\end{split}\label{main}
\end{equation}


\begin{itemize}
%\item $\Delta_N$: any function of the number of samples, $N$. For example, $\Delta_N = \frac{1}{N}$.
\item $a(V_t; \bm{\theta})$: Drift function 
\item $b (V_t; \bm{\theta} )$: Diffusion function 
\item $\bm{\theta}$: a vector of parameters
\item $dW_t$: Standard Wiener random process.
\end{itemize}
\end{frame}



\begin{frame}\frametitle{ Physical  Restriction }
It is necessary to keep  the power generation forecast plus its errors  inside the range $[0,1]$ where we have normalized against the power generating capacity of the power plant.  To ensure that, we choose our model to have zero diffusion near the boundaries. Hence, we  formulate the following diffusion  term, 
\begin{equation*}
b(V_t; \bm{\theta}) =  \sqrt{2 \theta \alpha p_t(1-p_t) X_t (1-X_t)}    = \sqrt{2 \theta \alpha p_t(1-p_t) (V_t +p_t ) (1-V_t-p_t)}  
\end{equation*}
It remains to control the drift term. As a first requirement, we need the drift term to be mean reverting to the power generation forecast. Thus, we formulate the following drift, 
\begin{equation}
a(V_t; \bm{\theta}) =  - \theta (X_t - p_t)= - \theta V_t 
\end{equation}
However, this does not track the rate of change of the forecast $\dot{p}$ which may results in $X_t$ exiting the interval $[0,1]$ when the forecast is a small $\delta$ distance away from the boundaries. 
\end{frame}


\begin{frame}\frametitle{ Physical  Restriction - Drift Near the Boundaries}
To see the issue, we switch to the point of view of the forecast prediction process $X_t$.  In this point of view, we have shown in our results that it is necessary to include derivative tracking in order to follow the power production forecast accurately.The derivative tracking model is given by,
\begin{equation}
\begin{split}
dX_t &= \hat{a}(X_t; \bm{\theta}) dt + \hat{b} (X_t; \bm{\theta} ) dW_t \quad t > 0 \\
X_0 & = x_0
\end{split}\label{main_X}
\end{equation}
where 
\begin{equation}
\begin{split}
\hat{a}(X_t; \bm{\theta}) &= \dot{p}_t + a(X_t; \bm{\theta}) =  \dot{p}_t - \theta \left(X_t - p_t\right)   \\
\hat{b}(X_t; \bm{\theta}) &=  b(X_t; \bm{\theta}) =  \sqrt{2 \theta \alpha p_t(1-p_t) X_t (1-X_t)}    \\
\end{split}
\end{equation}
It is clear in this point of view to see that  term $\dot{p}_t $ is not controlled to maintain that $X_t$ stays inside the  range $[0,1]$.
\end{frame}


\begin{frame}\frametitle{ Physical  Restriction - Drift Near the Boundaries}
We have that the first moment of the process $X_t$ is given by the following ODE, 
\begin{equation}
\dot{m}_1 (t) = \frac{d \E[ X_t] }{d t} = \dot{p}_t- \theta \E [ X_t ]  + \theta p_t =   \dot{p}_t- \theta m(t) + \theta p_t
\end{equation}
Re-arranging, 
\begin{equation}
\frac{\dot{m}_1 (t)}{\theta}  + m(t)=   \frac{\dot{p}_t}{\theta}  + p_t 
\end{equation}
Note that by setting $\hat{a}(X_t; \bm{\theta}) =0$, we find that  $ \frac{\dot{p}_t}{\theta}  + p_t $ is also a line of zero drift or a line of mean stationarity 
Thus, must have that, 
\begin{equation}
 0 \leq  \frac{\dot{p}_t}{\theta}  + p_t  \leq 1
\end{equation}
to keep the process $X_t$ inside the interval $[0,1]$.
The condition  can be rewritten as, 
\begin{equation}
\frac{- |\dot{p_t}|}{p_t} \leq \theta \leq \frac{|\dot{p}_t|}{1- p_t}
\end{equation}
\end{frame}


\begin{frame}\frametitle{ Physical  Restriction - Drift Near the Boundaries}
To enforce the condition, 
\begin{equation}
 0 \leq  \frac{\dot{p}_t}{\theta}  + p_t  \leq 1
 \label{cond_drift}
\end{equation}
We  define an adjusted drift $\theta = \theta_0 f( p_t, \dot{p}_t) $  which satisfies the condition when
\begin{equation}
\theta = \theta_0 f( p_t, \dot{p}_t) \leq \frac{|\dot{p}_t|}{\min (p_t, 1-p_t)} 
\end{equation}

Thus we choose $\theta$ such that,
\begin{equation}
\theta = \max \left( \theta_0 \ , \ \frac{|\dot{p}_t|}{\min (p_t, 1-p_t)}  \right )
\end{equation}

\end{frame}


\begin{frame}\frametitle{ Base Model }
We combine  the previous results to obtain the following SDE model, 
\begin{equation}
\begin{split}
dV_t &= a(V_t; \bm{\theta}) dt + b (V_t; \bm{\theta} ) dW_t \quad t > 0 \\
V_0 & = v_0
\end{split}\label{main}
\end{equation}


\begin{itemize}
%\item $\Delta_N$: any function of the number of samples, $N$. For example, $\Delta_N = \frac{1}{N}$.
\item $a(V_t; \bm{\theta})$: Drift function 
\item $b (V_t; \bm{\theta} )$: Diffusion function 
\item $\bm{\theta}$: a vector of parameters
\item $dW_t$: Standard Wiener random process.
\end{itemize}
\end{frame}



\begin{frame}
In order to respect the physical restrictions discussed above, we prescribe the following specifications
\begin{equation}
\begin{split}
a(V_t; \bm{\theta}) &= - \theta V_t \\
b(V_t; \bm{\theta}) &=\sqrt{2 \theta \alpha p_t(1-p_t) (V_t +p_t ) (1-V_t-p_t)}  \\
\end{split}
\end{equation}

with
\begin{equation}
\theta = \max \left( \theta_0 \ , \ \frac{|\dot{p}_t|}{\min (p_t, 1-p_t)}  \right )
\end{equation}

Where 
\begin{itemize}
\item $p_t$: Numerical wind power generation forecast.
\item $\theta >0$: Mean reversion parameter.
\item $\alpha>0$: Variability parameter.
\end{itemize}
\end{frame}


\begin{frame}\frametitle{ Implementation } 

\begin{itemize}
\item We linearly interpolate the forecast and data. Note that higher-order of interpolation cause the forecast and data to escape the interval $[0,1]$.
\item Extrapolate one hour backwards and take that as the new initial point.
\item We find the first integer $\kappa$ such that the line of mean stationarity $ \frac{\dot{p}_t}{\theta}  + p_t$ is inside the range $[0,1]$. 
\item Simulate each forecast using the modified $\theta = \theta_0 f(p_t)$ with its path specific constant $\kappa$.
\item Obtain the empirical  confidence intervals of each forecast.
\end{itemize}

\end{frame}


\begin{frame}\frametitle{ Results } 

See attached two PDF files, the file 6hr.pdf covers the first six hours while 72hr.pdf covers 72 hours.

\end{frame}



\againframe{guide}



\end{document}






























