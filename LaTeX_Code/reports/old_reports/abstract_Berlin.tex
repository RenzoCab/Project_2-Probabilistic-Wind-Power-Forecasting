\documentclass[a4paper, 12pt]{article}
\usepackage[utf8]{inputenc}
\usepackage[activeacute, english]{babel}
\usepackage{graphicx}
\usepackage{wrapfig}
\usepackage{amsmath}
\usepackage{bbm}
\usepackage{mathtools}
\usepackage{bm}
\usepackage{amssymb}
\usepackage{amsthm}
\usepackage{amsmath}
\usepackage{mathrsfs}
\usepackage{tikz}
\usepackage{float}
\usepackage{algpseudocode}
\usepackage{caption}
\usepackage{subcaption}

\newcommand\dist{\stackrel{\mathclap{\footnotesize\mbox{d.}}}{=}}
\usepackage[spanish,onelanguage,ruled,lined, linesnumbered]{algorithm2e}
\usepackage[spanish,onelanguage]{algorithm2e}
\newtheorem{theorem}{Teorema}[section]
\newtheorem{lemma}[theorem]{Lemma}
\newtheorem{proposition}[theorem]{Proposición}
%\newenvironment{proof}[1][Demostración]{\begin{trivlist}
%		\item[\hskip \labelsep {\bfseries #1}]}{\end{trivlist}}
\newenvironment{definition}[1][Definición]{\begin{trivlist}
		\item[\hskip \labelsep {\bfseries #1}]}{\end{trivlist}}
\newenvironment{example}[1][Ejemplo]{\begin{trivlist}
		\item[\hskip \labelsep {\bfseries #1}]}{\end{trivlist}}
\newenvironment{remark}[1][Nota]{\begin{trivlist}
		\item[\hskip \labelsep {\bfseries #1}]}{\end{trivlist}}
\newcommand{\negr}[1]{\textit{\textbf{#1}}}

\title{On the Uncertainty of Wind Power Forecasts \\ Abstract}

\author{Waleed Alhaddad \quad  Ra\'ul Fidel Tempone \\ CEMSE, KAUST}

\date{}
\begin{document}
\maketitle

\section{Motivation}

The adoption of renewable energy is accelerating due to technological advancements. However, the power generated by renewables is difficult to predict and thus such predictions require advanced uncertainty quantification techniques to ensure the bare minimum of meeting electricity power demands. This is also of importance for energy trading and insurance, that it is to deliver power in an optimal cost-effective way to adhere to a variety of financial contracts. Uncertainty quantification is key in the design  financial contracts and taking investment decisions in the energy sector. Additionally, it assists in the placement and design of renewable power plants.

\section{Introduction}

	We propose a model to simulate and quantify uncertainties in wind power generation forecasts. This model is based on Stochastic Differential Equations (SDEs) whose time-dependent parameters are inferred using optimization techniques of an associated approximate Likelihood function. Through continuous optimization, we find the optimal parameters of an unbounded convex problem with convex constraints.

	We are able to simulate and quantify uncertainties in a variety of wind power generation forecasts while taking into account the skewness of the errors. This method is non-intrusive and is independent of forecasting technology or future developments. Through optimization, we update and tune the parameters of our SDE as we receive new sets of observations and their associated forecasts. Additionally, we are able to compare in a quantitative manner the different forecast technologies and how they behave in different real-world scenarios. This model is to be extended to the uncertainty quantification of other power generation sources such as the uncertainties in the power generation of solar power plants. This introduces new challenges in terms of optimization and modeling. Finally, we apply our model to synthetic and real wind power generation data for benchmarking. We apply the model to Uruguayan wind power production as determined by historical data and corresponding numerical forecasts for the period of March 1 to May 31, 2016.


\section{Model}
Wind power generation forecasts have certain properties that need to be satisfied. One such particularity is that the forecasts are limited to the maximum power capabilities of the wind farm. Thus a proposed model for the uncertainties of the forecast must be controlled and restricted to a certain range. Due to this requirement, we propose an SDE whose solution is a controlled stochastic process that is contained inside a given range. Another challenge is when the forecast approaches the boundaries of this power production range. In this case, the errors of the forecast are compressed into a skew or asymmetric distribution. This renders the Gaussian approximations in such situations as unrealistic. Therefore, we propose a new approximate model based on a skew stochastic process.

Another particularity of forecast uncertainties is that the validity of the forecast lessens as time evolves. Usually, such forecasts are prepared on a periodic basis and their uncertainty grows in time. This is taken into account through the mean reversion and variability of the stochastic process associated with the SDE. This behavior is learned through the time-dependent parameters of the SDE.

\section{Conclusion}
We were able to simulate and quantify the uncertainties for a diverse range of forecasts based on the Uruguayan wind power production as determined by historical data and corresponding numerical forecasts for the period of March 1 to May 31, 2016. We compare our results to similar works in the literature. And we highlight the importance of asymmetry and the avoidance of phase shift artifacts which we were able to overcome.
 

%\subsection{Model}
%\begin{thebibliography}{99}
%	\bibitem{apo} Apostol, T. (1974). \emph{Mathematical Analysis} (2nd ed.),  Addison-Wesley. pp.218-230
%	\bibitem{E} Cabaña, E. (2012). \emph{Introducción a los
%		procesos estocásticos}, Notas
%	para el Curso de la Licenciatura en Estadística, Universidad de la República. pp.98-115
%\end{thebibliography}


\end{document}

