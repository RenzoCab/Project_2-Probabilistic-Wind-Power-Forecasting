\documentclass[aspectratio=169]{beamer}\usepackage[utf8]{inputenc}
\usepackage[english]{babel}
\usepackage{color}
\usepackage{amsmath,mathtools}
\usepackage{mathptmx}
\usepackage[11pt]{moresize}
\setbeamertemplate{navigation symbols}{}
\setbeamersize{text margin left=5mm,text margin right=5mm}
\usepackage{wrapfig}
\usepackage{bbm}
\usepackage{xcolor}
\usepackage{tabularx}
\usepackage{bm}
\usepackage{lmodern}


\newcommand{\R}{\mathbb{R}}
\newcommand{\E}{\mathbb{E}}
\newcommand{\N}{\mathbb{N}}
\newcommand{\Z}{\mathbb{Z}}
\newcommand{\V}{\mathbb{V}}
\newcommand{\Q}{\mathbb{Q}}
\newcommand{\K}{\mathbb{K}}
\newcommand{\C}{\mathbb{C}}
\newcommand{\T}{\mathbb{T}}
\newcommand{\I}{\mathbb{I}}

\setbeamertemplate{caption}[numbered]

\title{On the Uncertainty of Wind Power Generation\\ continuous report}
\subtitle{ Waleed Alhaddad}

\begin{document}
\setbeamercolor{background canvas}{bg=blue!1}


\begin{frame}
\titlepage
\end{frame}

\begin{frame}[label=guide]\frametitle{ Reader's Guide }
List of changes in this iteration:
\begin{itemize}
\item Removed the term $p(1-p)$ from the model.
    \begin{itemize}
        \item derived the new ODE for the moments.
        \item simulated and built the confidence intervals.
        \end{itemize}
\item Non-academic poster prepared.
\end{itemize}
Next steps:
\begin{itemize}
\item The french data set
    \begin{itemize}
        \item Organize and clean.
        \item Normalize by the changing installed capacity.
        \item Obtain the optimal parameters.
        \item generate paths and build the confidence bands.
        \item consider seasonality and human intervention.
    \end{itemize}
\item Check the performance of the optimization.
\item simultaneous versus point-wise confidence bands.
\end{itemize}
\small{
Note :
\begin{itemize}
\item Green slides: possible future extensions
\item Red slides: notes to be removed
\end{itemize}
}
\end{frame}

\begin{frame}\frametitle{ Moments of the process }
After removing the term $p_t(1-p_t)$ from the model, the first moment is unaffected and the second moment of the process is now given by,
\begin{equation*}
\frac{d m_2 (t)}{dt} = 2 \right( - m_2(t)\theta(1+\alpha) + \alpha\theta m_1(t)( (1-2p_t) + p_t(1-p_t)) \left)
\end{equation*}
\end{frame}


\begin{frame}\frametitle{ Moments of the process after Lamperti transform }
After removing the term $p_t(1-p_t)$ from the model, the moments of the process are now given by,


\begin{equation*}
m_1 (t_2) = \arcsin \left(  e^{-\int_{t_1}^{t_2} \theta_t (1-2 \alpha) \ dt  } \left( \int_{t_1}^{t_2} \theta_s (2p_s - 1)e^{ \int_{t_1}^{s} \theta_t (1-2 \alpha) \ dt  } \ ds + \sin (z_{t_1})\right)  \right)
\end{equation*}

\begin{equation*}
m_2 (t_2) =  2\alpha \int_{t_1}^{t_2}  \theta_t  \ dt + m_2(t_1)
\end{equation*}

\end{frame}

\againframe{guide}



\end{document}
