\documentclass[11pt]{article}
\usepackage[utf8]{inputenc}
\usepackage{amsmath}    % need for subequations
\usepackage{graphicx}   % need for figures
\usepackage{verbatim}   % useful for program listings
\usepackage{color}      % use if color is used in text
%\usepackage{subfigure}  % use for side-by-side figures
%\usepackage{hyperref}   % use for hypertext links, including those to external documents and URLs

%\usepackage[]{showkeys}

%%%%% Comments %%%%%
\usepackage{todonotes}
\usepackage{ulem}
%%%%% Comments %%%%%

\usepackage{amssymb}
\usepackage{amsthm}
\usepackage{fancybox}

% don't need the following. simply use defaults
\setlength{\baselineskip}{16.0pt}    % 16 pt usual spacing between lines

\setlength{\parskip}{3pt plus 2pt}
\setlength{\parindent}{20pt}
\setlength{\oddsidemargin}{0.5cm}
\setlength{\evensidemargin}{0.5cm}
\setlength{\marginparsep}{0.75cm}
\setlength{\marginparwidth}{2.5cm}
\setlength{\marginparpush}{1.0cm}
\setlength{\textwidth}{150mm}
\theoremstyle{definition}
\newtheorem{Def}{Definition}[section]
\newtheorem{Eg}{Example}[section]
\newtheorem{Rm}{Remark}[section]
\newtheorem{Ex}{Exercise}[section]

%\theoremstyle{plain}
\newtheorem{Prop}[Def]{Proposition}
\newtheorem{Lem}[Def]{Lemma}
\newtheorem{Thm}[Def]{Theorem}
\newtheorem{Cor}[Def]{Corollary}
\newtheorem{Rem}[Def]{Remark}
\newtheorem{Ass}[Def]{Assumption}
\newtheorem{Met}[Def]{Method}
\newtheorem{Res}[Def]{Result}

\newcommand{\p}{\mathbb{P}}
\newcommand{\e}{\mathbb{E}}
\newcommand{\real}{\mathbb{R}}
\newcommand{\n}{\mathbb{N}}
\newcommand{\comp}{\mathbb{C}}
\newcommand{\z}{\mathbb{Z}}

%\usepackage{slashbox}

%\begin{comment}
%\pagestyle{empty} % use if page numbers not wanted
%\end{comment}

% above is the preamble

\allowdisplaybreaks

\begin{document}	
\section{The model}
For a time horizon $T>0$ and  a parameter $\alpha > 0$ and $(\theta_t)_{t\in[0,T]}$ a positive deterministic  function,  let us consider the model  given by
\begin{align}\label{eq:model}
dX_t&= \big(\dot p_t - \theta_t(X_t-p_t)  \big)dt  +\sqrt{2\alpha \theta_tX_t(1-X_t)} dW_t\quad t\in[0,T]\\
X_0&=x_0\in[0,1],\notag
\end{align}
where $(p_t)_{t\in[0,T]}$ denotes the prediction function that satisfies $0\le p_t\le 1$ for all $t\in[0,T]$. This prediction function is assumed to be a smooth function of the time so that 
$$\sup_{t\in[0,T]}\bigl( |p_s| + |\dot p_s|\big) <+\infty .$$
The following proofs are based on standard arguments for stochastic processes that can be found e.g. in  Alfonsi \cite{Alf} and Karatzas and Shreve \cite{KarShr} that we adapted to the setting of our model \eqref{eq:model}.
\begin{Thm}\label{thm:exun}
Assume that    
\begin{equation}\label{Assumption:1}
\forall  t\in[0,T],\;\; 0\le \dot p_t +\theta_tp_t\le \theta_t, \;\;\mbox{ and }\;\;
\sup_{t\in[0,T]}|\theta_t|<+\infty\tag{A}. 
\end{equation}
Then, there is a unique strong solution to \eqref{eq:model} s.t.  for all $t\in[0,T]$, $X_t\in[0,1]$ a.s.
\end{Thm}
\begin{proof}
Let us first consider the following SDE for $t\in[0,T]$
\begin{equation}\label{eq:eds1}
X_t=x_0+  \int_0^t\big(\dot p_s - \theta_s(X_s-p_s)  \big)ds  + \int_0^t\sqrt{2\alpha \theta_s|X_s(1-X_s)|} dW_s, \quad x_0>0.
\end{equation}
According to Proposition 2.13, p. 291 of \cite{KarShr}, under assumption  \eqref{Assumption:1} there is a unique strong solution $X$ to \eqref{eq:eds1}. Moreover, as the diffusion coefficient is of linear growth, we have  for all $p>0$ 
\begin{equation}\label{prop:fm}
\mathbb E[ \sup_{t\in[0,T]}|X_t|^p]<\infty.
\end{equation}
Then, it remains to show that for all $t\in[0,T]$, $X_t\in[0,1]$ a.s. For this aim, we need to use the so-called Yamada function $\psi_n$ that is a $\mathcal C^2$ function that satisfies a bench of useful properties:
\begin{align*}
&|\psi_n(x)|\underset{n\rightarrow+\infty}{\rightarrow}|x|, \;\; x{\psi'}_n(x)\underset{n\rightarrow+\infty}{\rightarrow}|x|, \;\; |\psi_n(x)|\wedge |x{\psi'}_n(x)| \le |x|\\
&{\psi'}_n(x)\le 1, \;\; \mbox{ and } {\psi''}_n(x)=g_n(|x|)\ge 0\;\; \mbox{ with } \;\; g_n(x)x\le \frac 2n\;\;  \mbox{ for all } x\in \mathbb R .
\end{align*}
See the proof of Proposition 2.13, p. 291 of \cite{KarShr} for the construction of such fucntion.
Applying Itô's formula we get
\begin{align*}
\psi_n(X_t)&=\psi_n(x_0) +\int_0^t {\psi'}_n(X_s)(\dot p_s + \theta_s p_s - \theta_sX_s  \big)ds + \int_0^t{\psi'}_n(X_s)\sqrt{2\alpha \theta_s|X_s(1-X_s)|} dW_s\\
&+ \alpha \int_0^t \theta_sg_n(|X_s|)|X_s(1-X_s)|ds.
\end{align*}
Now, thanks to  \eqref{Assumption:1}, \eqref{prop:fm} and to the above properties of $\psi_n$ and $g_n$, we get
$$
\mathbb E[\psi_n(X_t)]\le \psi_n(x_0) +\int_0^t \big(\dot p_s + \theta_sp_s -  \theta_s \mathbb E[{\psi'}_n(X_s)X_s] \big)ds + \frac{2\alpha}{n}\int_0^t \theta_s\mathbb E |1-X_s|ds.
$$
Therefore, letting $n$ tends to infinity we use Lebesgue's theorem to get
$$
\mathbb E[|X_t|]\le x_0 +\int_0^t \big(\dot p_s + \theta_sp_s -  \theta_s \mathbb E|X_s| \big)ds.
$$
Besides, taking the expectation of \eqref{eq:eds1}, we get
$$
\mathbb E X_t=x_0+  \int_0^t\big(\dot p_s +\theta_sp_s - \theta_s \mathbb EX_s  \big)ds
$$
and thus we have 
$$
\mathbb E[|X_t| -X_t ]\le \int_0^t \theta_s \mathbb E[|X_s| - X_s ]ds.
$$
Then, Gronwall's lemma gives us $\mathbb E[|X_t|]=\mathbb E X_t$ and thus for any $t\in[0,T]$ $X_t\ge0$ a.s. The same arguments work to prove that  for any $t\in[0,T]$ $Y_t:=1-X_t\ge0$  a.s.  since the process $(Y_t)_{t\in[0,T]}$ is solution to 
$$
dY_t= \big( \theta_t(1-p_t) -\dot p_t - \theta_tY_t  \big)dt  -\sqrt{2\alpha \theta_tY_t(1-Y_t)} dW_t,
$$
Then similarly, we need to assume that $\dot p_t +\theta_tp_t\ge 0$. This completes the proof.
\end{proof}

\begin{Thm}
Assume that assumptions of Theorem \ref{thm:exun} hold with $x_0\in]0,1[$.
Let $\tau_0:=\inf \{t\in[0,T],\; X_t=0\}$ and  $\tau_1:=\inf \{t\in[0,T],\; X_t=1\}$ with the convention that $\inf\emptyset=+\infty$. Assume in addition that
\begin{equation}\label{Assumption:2}
0<\alpha < \frac 12\quad \mbox{ and }\quad \forall  t\in[0,T],\;\; \alpha\theta_t\le \dot p_t +\theta_tp_t \le (1-\alpha)\theta_t \tag{B}. 
\end{equation}
 Then, $\tau_0=\tau_1=+\infty$ a.s.
\end{Thm}
\begin{proof}
For $t\in[0,\tau_0[$, we have 
$$
\frac{dX_t}{X_t}= \frac{\dot p_t +\theta_t p_t}{X_t} - \theta_tdt  +\sqrt{\frac{2\alpha \theta_t(1-X_t)}{X_t}} dW_t 
$$  
so that
$$
X_t=x_0\exp\Big(\int_0^t \frac{\dot p_s +\theta_s(p_s-\alpha)}{X_s}ds-(1-\alpha)\int_0^t\theta_sds + M_t\Big),
$$
where $M_t=\int_0^t\sqrt{\frac{2\alpha \theta_s(1-X_s)}{X_s}} dW_s$ is a continuous martingale. Then as for all $t\in[0,T]$, we have $\dot p_t +\theta_t(p_t-\alpha)\ge0$, we deduce that
$$
X_t\ge x_0\exp\Big(-(1-\alpha)\int_0^t\theta_sds + M_t\Big).
$$
By way of contradiction let us assume that  $\{\tau_0<\infty\}$, then letting $t\to \tau_0$ we deduce that $\lim_{t\to \infty} \mathbf 1_{\{\tau_0<\infty\}}M_{t\wedge \tau_0}=\mathbf -1_{\{\tau_0<\infty\}}\infty$ a.s. This leads to a contradiction since we know that continuous martingales likewise the Brownian motion cannot converge  almost surely to $+\infty$ or $-\infty$. It follows that $\tau_0=\infty$ almost surely. Next, recalling that  the process $(Y_t)_{t\geq 0}$  given by $Y_t=1-X_t$ is solution to 
$$
dY_t= \big( \theta_t(1-p_t) -\dot p_t - \theta_tY_t  \big)dt  -\sqrt{2\alpha \theta_tY_t(1-Y_t)} dW_t 
$$
we deduce using similar arguments as above 
$\tau_1=\infty$ a.s. provided that $\theta_t(1-p_t) -\dot p_t -\alpha \theta_t\ge 0$.
 \end{proof}
\textcolor{red}{When using Lamperti, the form }
\begin{thebibliography}{00}

\bibitem{Alf} Alfonsi, A. (2015) Affine diffusions and related processes: simulation, theory and applications. {\it  Springer, Cham; Bocconi University Press, Milan, 2015}. 
\bibitem{KarShr} Karatzas, I. and Shreve, S. E. (1991),  Brownian motion and stochastic calculus {\it Springer-Verlag, New York}.

\end{thebibliography}
\end{document}