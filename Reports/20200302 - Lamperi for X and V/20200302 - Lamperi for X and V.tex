\PassOptionsToPackage{table}{xcolor}
\documentclass[aspectratio=169]{beamer}\usepackage[utf8]{inputenc}
\usepackage{lmodern}
\usepackage[english]{babel}
\usepackage{color}
\usepackage{amsmath,mathtools}
\usepackage{booktabs}
\usepackage{mathptmx}
\usepackage[11pt]{moresize}
\usepackage{hyperref}
\usepackage{commath}
\usepackage{bm}
\usepackage{subfigure}
\usepackage{siunitx}

\setbeamertemplate{navigation symbols}{}
\setbeamersize{text margin left=5mm,text margin right=5mm}
\setbeamertemplate{caption}[numbered]
\addtobeamertemplate{navigation symbols}{}{
\usebeamerfont{footline}
\usebeamercolor[fg]{footline}
\hspace{1em}
\insertframenumber/\inserttotalframenumber}

\newcommand{\R}{\mathbb{R}}
\newcommand{\E}{\mathbb{E}}
\newcommand{\N}{\mathbb{N}}
\newcommand{\Z}{\mathbb{Z}}
\newcommand{\V}{\mathbb{V}}
\newcommand{\Q}{\mathbb{Q}}
\newcommand{\K}{\mathbb{K}}
\newcommand{\C}{\mathbb{C}}
\newcommand{\T}{\mathbb{T}}
\newcommand{\I}{\mathbb{I}}
\DeclareMathOperator{\sign}{sign}

\title{Lamperti Transform for the processes X and V}
\subtitle{Renzo Miguel Caballero Rosas}

\begin{document}

\begin{frame}
\titlepage
\end{frame}


\setbeamercolor{background canvas}{bg=white!10}
\begin{frame}\frametitle{New model for the SDE: $\theta_t=\theta_0$ in the diffusion}

\begin{enumerate}

\item[$X_t$:] $\dif X_t=\left(\dot p_t-\theta_t(X_t-p_t)\right)\dif t+\sqrt{2\alert{\theta_0}\alpha X_t(1-X_t)}\dif W_t$
\item[$V_t$:] $\dif V_t=-\theta_tV_t\dif t+\sqrt{2\alert{\theta_0}\alpha(V_t+p_t)(1-V_t-p_t)}\dif W_t$

\end{enumerate}
Lamperti transform for $V_t$:
\begin{equation*}
\begin{split}
\psi(V_t,t)=\int\frac{1}{\sqrt{2\alert{\theta_0}\alpha(u+p_t)(1-u-p_t)}}\dif u\Bigg|_{u=V_t}&=-\sqrt{\frac{2}{\alpha\alert{\theta_0}}}\arcsin\left(\sqrt{1-V_t-p_t}\right),\\
&=-\sqrt{\frac{2}{\alpha\alert{\theta_0}}}\arcsin\left(\sqrt{1-X_t}\right).
\end{split}
\end{equation*}

We can see that for every $t=t^*$, the primitive function of $\frac{1}{\sigma(v,t^*)}$ is well defined for all $v\in\left[-p(t^*),1-p(t^*)\right]\subset[-1,1]$ (recall $v=x-p_t$, and $x\in[0,1]$; then, when $x=0$ and $x=1$, we have that $v=-p$ and $x=1-p$, respectively).

\end{frame}


\setbeamercolor{background canvas}{bg=white!10}
\begin{frame}\frametitle{Identities for the Lamperti transform of $V_t$:} \label{Q1}

\begin{itemize}
\item $\psi(V_t,t)=-\sqrt{\frac{2}{\alpha\alert{\theta_0}}}\arcsin(\sqrt{1-V_t-p_t})$.
\item $\psi_v(V_t,t)=\frac{1}{\sigma(V_t,t)}=\frac{1}{\sqrt{2\alpha\alert{\theta_0}(V_t+p_t)(1-V_t-p_t)}}$.
\item $\psi_{vv}(V_t,t)=\frac{\dif}{\dif v}\left[\frac{1}{\sigma(V_t,t)}\right]=-\frac{\sigma_v(V_t,t)}{\sigma^2(V_t,t)}=-\frac{1}{\sigma^2(V_t,t)}\cdot\sqrt{\frac{\alpha\alert{\theta_0}}{2}}\frac{1-2V_t-2p_t}{\sqrt{(V_t+p_t)(1-V_t-p_t)}}$.
\item $\psi_t(V_t,t)=\frac{\dot{p}_t}{\sqrt{2\alpha\alert{\theta_0}(V_t+p_t)(1-V_t-p_t)}}$.
\end{itemize}
\quad\\
\quad\\
Recall $V_t=X_t-p_t$. Then $\psi_v$, $\psi_{vv}$, and $\psi_t$ are not defined when $X_t=0$ or $X_t=1$. However, this only happens in the boundary of the domain $(0,1)$.\\
\quad\\
\alert{Can we apply It\^o's lemma to $\psi$? Maybe as the singularities are in the boundary, it is possible despite that we are not strictly in the hypothesis of the lemma.}
\end{frame}


\setbeamercolor{background canvas}{bg=white!10}
\begin{frame}\frametitle{SDE for $Z_t=\psi(V_t,t)$: ({\color{green}Verified with Mathematica})}

By It\^o's lemma, if $\psi(v,t)$ is $C^2([-p_t,1-p_t])$ for $v$ and $C^1([0,T])$ for $t$, then:
\begin{equation*}
\dif Z_t=\left({\color{blue}\psi_t}+\psi_v\cdot f+\alert{\frac{1}{2}\psi_{vv}\cdot\sigma^2}\right)\dif t+\psi_v\cdot\sigma\dif W_t.
\end{equation*}
If we substitute the terms related with $\psi(V_t,t)$ from slide (\ref{Q1}), we have
\begin{equation*}
{\footnotesize
\begin{split}
\dif Z_t=&\Bigg[{\color{blue}\frac{\dot{p}_t}{\sqrt{2\alpha\theta_0(V_t+p_t)(1-V_t-p_t)}}}\\
&-\frac{\theta_tV_t}{\sqrt{2\alpha\theta_0(V_t+p_t)(1-V_t-p_t)}}\alert{-\frac{1}{2}\sqrt{\frac{\alpha\theta_0}{2}}\frac{1-2V_t-2p_t}{\sqrt{(V_t+p_t)(1-V_t-p_t)}}}\Bigg]\dif t+1\cdot \dif W_t.
\end{split}}
\end{equation*}
Recall $\alert{Z_t=-\sqrt{\frac{2}{\alpha\theta_t}}\arcsin\left(\sqrt{1-V_t-p_t}\right)}$, where $Z_t\in\Big[-\frac{\pi}{\sqrt{2\alpha\theta_t}},{\color{black}0\Big]}$.
\end{frame}


\setbeamercolor{background canvas}{bg=white!10}
\begin{frame}\frametitle{SDE for $Z_t=\psi(V_t,t)$: ({\color{green}Computed with Mathematica})} \label{Q2}

\begin{equation*}
\dif Z_t=\underbrace{\left[\frac{\alpha\theta_0\cos(Z_t\sqrt{2\alpha\theta_0})-\theta_t\cos(Z_t\sqrt{2\alpha\theta_0})+2\theta_tp_t+2\dot{p}_t-\theta_t}{\sqrt{\alpha\theta_0}\sqrt{1-\cos(2Z_t\sqrt{2\alpha\theta_0})}}\right]}_{f(Z_t,t)}\dif t+1\cdot\dif W_t.
\end{equation*}\\
\begin{equation*}
\lim_{z\to0^-}f(z,t)=\infty\times\left[\frac{\sign\left(2\theta_tp_t+2\dot{p}_t+\alpha\theta_0-2\theta_t\right)}{\sign(\alpha)\sign(\theta_0)}\right].
\end{equation*}
\begin{equation*}
\lim_{z\to\left[\frac{-\pi}{\sqrt{2\alpha\theta_0}}\right]^+}f(z,t)=\infty\times\left[\frac{\sign\left(2\theta_tp_t+2\dot{p}_t-\alpha\theta_0\right)}{\sign(\alpha)\sign(\theta_0)}\right].
\end{equation*}
We want to find the correct conditions for $\theta_t$.\\
\quad\\
{\small To simplify the SDE, Mathematica has used: $\sin^2(x)-\sin^4(x)=\sin^2(x)\cos^2(x)=\frac{1}{4}\sin^2(2x)=\frac{1}{8}(1-\cos(4x))$.}

\end{frame}


\setbeamercolor{background canvas}{bg=white!10}
\begin{frame}\frametitle{Limit when $z\to0^-$:}\label{C1}

Recall we have a bijective mapping $Z_t({\color{blue}[0},\alert{1]})={\color{blue}\Big[\frac{-\pi}{\sqrt{2\alpha\theta_0}}},\alert{0\Big]}$. This helps the intuition, as when $X_t=1$, we expect the diffusion to be negative, and $z\to0^-$ is equivalent to $x\to1^-$.\\
\quad\\
We want $\lim_{z\to0^-}f(z,t)$ to be $-\infty$ or zero, so we do not escape from $z=0$ to $z>0$ ($x=1$ to $x>1$). Then, we need $\alpha\theta_0-2\theta_t+2\theta_tp_t+2\dot{p}_t\leq0$. Then:
\begin{itemize}

\item If $p_t<1$, we have that ${\color{orange}\theta_t\geq\frac{\alpha\theta_0+2\dot{p}_t}{2(1-p_t)}}$.

\end{itemize}

\end{frame}


\setbeamercolor{background canvas}{bg=white!10}
\begin{frame}\frametitle{Limit when $z\to\left[\frac{-\pi}{\sqrt{2\alpha\theta_0}}\right]^+$:}\label{C2}

Recall we have a bijective mapping $Z_t({\color{blue}[0},\alert{1]})={\color{blue}\Big[\frac{-\pi}{\sqrt{2\alpha\theta_0}}},\alert{0\Big]}$. This helps the intuition, as when $X_t=0$, we expect the diffusion to be positive, and $z\to\frac{-\pi}{\sqrt{2\alpha\theta_0}}^+$ is equivalent to $x\to0^+$.\\
\quad\\
We want $\lim_{z\to\left[\frac{-\pi}{\sqrt{2\alpha\theta_0}}\right]^+}f(z,t)$ to be $+\infty$ or zero, so we do not escape from $z=\frac{-\pi}{\sqrt{2\alpha\theta_t}}$ to $z<\frac{-\pi}{\sqrt{2\alpha\theta_t}}$ ($x=0$ to $x<0$). Then, we need $2\theta_tp_t+2\dot{p}_t-\alpha\theta_0\geq0$. Then:
\begin{itemize}

\item If $p_t>0$, we have that ${\color{orange}\theta_t\geq\frac{\alpha\theta_0-2\dot{p}_t}{2p_t}}$.

\end{itemize}

\end{frame}


\setbeamercolor{background canvas}{bg=white!10}
\begin{frame}\frametitle{Controlled drift:}

From both {\color{orange}orange} conditions in slides (\ref{C1}) and (\ref{C2}), we create a more restrictive condition:
\begin{equation*}
\max\left({\color{orange}\frac{\alpha\theta_0+2\dot{p}_t}{2(1-p_t)}},{\color{orange}\frac{\alpha\theta_0-2\dot{p}_t}{2p_t}}\right)\leq\frac{\alpha\theta_0+|2\dot{p}_t|}{2\min(1-p_t,p_t)}.
\end{equation*}
Then, we choose
\begin{equation}
\theta_t=\max\left(\theta_0,\frac{\alpha\theta_0+|2\dot{p}_t|}{2\min(1-p_t,p_t)}\right).
\label{NC}
\end{equation}
\quad\\
Recall that in the paper, we start by choosing $\theta_t=\max\left(\theta_0,\frac{|\dot{p}_t|}{\min(1-p_t,p_t)}\right)$. Our new condition (\ref{NC}) is slightly more restrictive.

\end{frame}


\setbeamercolor{background canvas}{bg=white!10}
\begin{frame}\frametitle{Limits when $p_t\approx0$ and $p_t\approx1$:}

It $p_t\approx0$, then $\theta_t=\frac{\alpha\theta_0+|2\dot{p}_t|}{2p_t}$, and we have the limit: $$\lim_{z\to\left[\frac{-\pi}{\sqrt{2\alpha\theta_0}}\right]^+}f(z,t)=\infty\times\sign(|\dot{p}_t|+\dot{p}_t).$$
As $p_t\approx0$, it is reasonable to assume $\dot{p}_t\geq0$. Then, the limit is $+\infty$.\\
\quad\\
It $p_t\approx1$, then $\theta_t=\frac{\alpha\theta_0+|2\dot{p}_t|}{2(1-p_t)}$, and we have the limit: $$\lim_{z\to0^-}f(z,t)=\infty\times\sign(\dot{p}_t-|\dot{p}_t|).$$
As $p_t\approx1$, it is reasonable to assume $\dot{p}_t\leq0$. Then, the limit is $-\infty$.

\end{frame}


\setbeamercolor{background canvas}{bg=white!10}
\begin{frame}\frametitle{Conditions summary:}

\begin{itemize}

\item First model: $\theta_t^{first}=\max\left(\theta_0,\frac{|\dot{p}_t|}{\min(1-p_t,p_t)}\right)$.
\item New condition from Lamperti: $\theta_t^{lamperti}=\max\left(\theta_0,\frac{\alpha\theta_0+|2\dot{p}_t|}{2\min(1-p_t,p_t)}\right)$. Notice $\theta_t^{first}\leq\theta_t^{lamperti}$.
\item Professor Kebaier's condition: $0<\alpha<1/2$, $\alpha<p_t<1-\alpha$, and $\theta_t^{kebaier}=\max\left(\theta_0,\frac{|\dot{p}|}{\min(p-\alpha,1-p-\alpha)}\right)$.

\end{itemize}
Now, given $p_t\approx\alpha$, we have that $\theta_t^{lamperti}=\theta_t^{kebaier}$ if $\theta_0=\frac{|2\dot{p}_t|}{p_t-\alpha}>0$. Then, which is more restrictive depends on $|\dot{p}_t|$ and $p_t$.\\
\quad\\
Now, given $p_t\approx1-\alpha$, we have that $\theta_t^{lamperti}=\theta_t^{kebaier}$ if $\theta_0=\frac{|2\dot{p}_t|}{1-p_t-\alpha}>0$. Then, which is more restrictive depends on $|\dot{p}_t|$ and $p_t$.\\
\quad\\
\alert{We can see that, we can have $\theta_t^{lamperti}>\theta_t^{kebaier}$ or $\theta_t^{lamperti}<\theta_t^{kebaier}$, depending on the values of $p_t$ and $\dot{p}_t$.}
\end{frame}


\setbeamercolor{background canvas}{bg=white!10}
\begin{frame}\frametitle{Particular questions:}

\begin{itemize}

\item In slide ({\color{blue}\ref{Q1}}), we can see that the Lamperti transform $\psi(v,t)$ has undefined partial derivatives when $x=0$, or $x=1$. This is a consequence of the singularities of $\frac{1}{\sigma(v,t)}$ when $x=0$, or $x=1$. \alert{What can we say about the SDE of $Z_t=\psi(V_t,t)$ in the sense of existence and unicity? Can we use It\^o's lemma considering that the singularities are in the boundary of the domain?}
\item In slide ({\color{blue}\ref{Q2}}), the limits for the drift when $Z_t$ touch the boundaries of its domain depend on $\alpha$. Then, the condition for $Z_t$ to stay always in $\Big[\frac{-\pi}{\sqrt{2\alpha\theta_0}},0\Big]$ also depends on $\alpha$. \alert{This is not intuitive because the condition for $X_t$ to be in $[0,1]$, and there is a bijective mapping between $X_t$ and $Z_t$, so they both should require the same condition.}

\end{itemize}

\end{frame}


\end{document}