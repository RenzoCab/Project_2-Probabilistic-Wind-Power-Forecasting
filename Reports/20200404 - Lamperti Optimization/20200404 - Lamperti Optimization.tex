\PassOptionsToPackage{table}{xcolor}
\documentclass[aspectratio=169]{beamer}\usepackage[utf8]{inputenc}
\usepackage{lmodern}
\usepackage[english]{babel}
\usepackage{color}
\usepackage{amsmath,mathtools}
\usepackage{booktabs}
\usepackage{mathptmx}
\usepackage[11pt]{moresize}
\usepackage{hyperref}
\usepackage{commath}
\usepackage{bm}
\usepackage{subfigure}
\usepackage{siunitx}

\setbeamertemplate{navigation symbols}{}
\setbeamersize{text margin left=5mm,text margin right=5mm}
\setbeamertemplate{caption}[numbered]
\addtobeamertemplate{navigation symbols}{}{
\usebeamerfont{footline}
\usebeamercolor[fg]{footline}
\hspace{1em}
\insertframenumber/\inserttotalframenumber}

\newcommand{\R}{\mathbb{R}}
\newcommand{\E}{\mathbb{E}}
\newcommand{\N}{\mathbb{N}}
\newcommand{\Z}{\mathbb{Z}}
\newcommand{\V}{\mathbb{V}}
\newcommand{\Q}{\mathbb{Q}}
\newcommand{\K}{\mathbb{K}}
\newcommand{\C}{\mathbb{C}}
\newcommand{\T}{\mathbb{T}}
\newcommand{\I}{\mathbb{I}}
\DeclareMathOperator{\sign}{sign}

\title{Lamperti Optimization}
\subtitle{Renzo Miguel Caballero Rosas}

\begin{document}

\begin{frame}
\titlepage
\end{frame}


\setbeamercolor{background canvas}{bg=white!10}
\begin{frame}\frametitle{Description:}

Let $\{\Delta V_i\}_{i=1}^n$ be the set of all error transitions, and $\psi(\theta_0,\alpha,\Delta V)$ the Lamperti transform. Notice that the Lamperti transitions $\{\Delta Z_i\}_{i=1}^n$ depend on $(\theta_0,\alpha)$ because $$\{\Delta Z_i\}_{i=1}^n=\psi(\theta_0,\alpha,\{\Delta V_i\}_{i=1}^n).$$
Then, if we compute $$\min_{(\theta_0,\alpha)}\mathbf{L}(\theta_0,\alpha,\{\Delta Z_i\}_{i=1}^n),$$
we are \alert{NOT} computing a MLE in the classical sense. However, we can try to find ${\color{blue}(\theta_0^*,\alpha^*)}$ such that $${\color{blue}(\theta_0^*,\alpha^*)}=\arg\min_{(\theta_0,\alpha)}\mathbf{L}\left(\theta_0,\alpha,\psi({\color{blue}\theta_0^*},{\color{blue}\alpha^*},\{\Delta V_i\}_{i=1}^n)\right).$$
In this point, the likelihood has a maximum for the data set corresponding to that point.

\end{frame}

\end{document}